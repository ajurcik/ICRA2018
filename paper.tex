%\documentclass[a4paper, 10pt]{IEEEtran}      % for icra/RAL final
\documentclass[letterpaper, 10 pt, conference]{ieeeconf} % initial submission to RAL/ICRA

\usepackage{graphicx}
\usepackage{booktabs}
%\usepackage{amsmath}
\usepackage{amsfonts}
\usepackage{subfig}
\usepackage[ruled,lined]{algorithm2e}
\usepackage{multirow}
%\usepackage[bookmarks=false]{hyperref}
%\usepackage{setspace}
\pdfminorversion=4

%\IEEEoverridecommandlockouts                              % This command is only
%\overrideIEEEmargins


\newcommand{\red}[1]{\textcolor{red}{#1}}
\newcommand{\tylde}{$\sim$}

%\hypersetup{colorlinks=true,linkcolor=black,citecolor=black}

\SetKwInput{KwData}{Global params.}

\title{Motion planning for assessing accessibility of protein tunnels}

\author{Vojt\v ech Von\' asek$^{1}$,
    Barbora Kozl\'\i kov\'a$^{2}$,
    Adam Jur\v{c}\'\i k$^{2}$,
    Martin Saska$^{1}$
\thanks{$^{1}$ Faculty of Electrical Engineering,  Czech Technical University in Prague, 
Technick\'a 2, 166 27 Prague, Czech Republic,
{\tt vonasek@labe.felk.cvut.cz}}
\thanks{$^{2}$        
Faculty of Informatics,   Masaryk University, 
Botanick\'a 68a, 602~00 Brno, Czech Republic,\\
The presented work has been supported by the Czech Science Foundation (GA{\v C}R) under research project No. 17-07690S.
Access to computing and storage facilities owned by parties and projects contributing to the National Grid Infrastructure MetaCentrum, provided under the programme "Projects of Large Infrastructure for Research, Development, and Innovations" (LM2010005), is greatly appreciated.}}


%D. Bedn\'{a}\v{r}:
%Faculty of Science, Masaryk University, Kamenice 5, 625 00 Brno, Czech Republic\\


\def\qrand{q_{rand}}
\def\qstart{q_{start}}
\def\qinit{\qstart}
\def\qgoal{q_{goal}}
\def\qnear{q_{near}}
\def\qnew{q_{new}}
\def\T{\mathcal{T}}

\def\C{\mathcal{C}}
\def\CF{\mathcal{C}_{free}}
\def\CFD{{\mathcal{C}^s_{free}}}
\def\dt{d_{tunnel}}
\def\da{d_{atom}}
\def\R{\mathbb{R}}
\def\QI{Q_{init}}
\def\RI{R_{init}}

\def\rv{R_{tunnel}}


\def\Imax{I_{max}} %max number of iterations of RRT-based planners

\def\dist{\mathrm{dist}}
\def\dists{\mathrm{dist}_{\mathrm{s}}}

\SetKw{return}{return}

\def\smin{s_{min}}
\def\smax{s_{max}}
\def\sdelta{s_{\Delta}}


\def\probe{r_{\mathrm{probe}}}
\def\Sprobe{S_{\mathrm{probe}}}

\def\gprobe{r_{\mathrm{out}}}
\def\Sgprobe{S_{\mathrm{out}}}


\def\CG{\mathcal{C}_{goal}}
\def\SB{\mathbf{S}_{blocking}}
\def\SS{\mathbf{S}}


%spacing for algorithm environment. 1.0 mean normal spacing
\def\gb{p_{tunnel}}

\def\L{L}
\def\LA{L_1}
\def\LB{L_2}

% ==================================================================================

\begin{document}

\maketitle
\thispagestyle{empty}
\pagestyle{empty}

%\frontmatter          % for the preliminaries


\begin{abstract}
Chemical interactions of proteins and other molecules (ligands) take place in actives sites that are usually deeply buried inside the proteins.
The active sites are accessible through one or more pathways (called tunnels).
The knowledge about a tunnel (e.g. it's bottleneck or physico-chemical properties) can help chemists to estimate possible interactions
between the protein and the ligand.
The tunnels have been traditionally computed using Voronoi diagrams, which is suitable only for small, spherical and rigid ligands.
We propose to analyze the accessibility of the active sites using sampling-based motion planning.
This requires to cope with narrow passages and to model the flexibility of the ligands.
To cope with the narrow passage problem, we generate the random samples around a Voronoi-based tunnel and we shrink
the atomic radii of the ligand.
The flexibility of the ligand is modeled using a fixed set of known conformations.
\end{abstract}


\section{Introduction \& Motivation}


\begin{figure}[b]
\centering
{\footnotesize
\renewcommand{\arraystretch}{0.1}
\renewcommand{\tabcolsep}{0pt}
\begin{tabular}{ccc}
\includegraphics[width=0.15\textwidth]{fig/motiv1} &
\includegraphics[width=0.17\textwidth]{fig/motiv2lab} &
\includegraphics[width=0.16\textwidth]{fig/motiv3}  \\
%\hbox{
%\vbox{
%\hbox{\includegraphics[width=0.25\textwidth]{fig/ta-1} }
%\hbox{\includegraphics[width=0.25\textwidth]{fig/ta-433}}
%} 
%}
Protein 1CQW & Active site & Detected tunnels \\ %& Example of a  \\
             &            & (orange)          %& trajectory
\end{tabular}
}
\caption{\label{fig::motiv}
    Tunnels in haloalkane dehalogenase with a possible trajectory of 1-chlorpropan ligand.
}
\end{figure}

The idea of sampling-based motion planning is to randomly sample the configuration space of the robot and connect the collision-free samples
into a graph structure (roadmap).
A path in the roadmap then corresponds to a motion in the workspace.
As the random samples are classified as free or non-free using collision detection, the sampling-based methods can found paths for robots of various shapes. 
Another advantage of the sampling-based planners is that they can cope with robots with many degrees of freedom (DOF).
The two most used sampling-based techniques are Probabilistic Roadmaps (PRM)~\cite{kavrakiForPP} and Rapidly Exploring Random Trees (RRT)~\cite{lavalleRRT}.
These planners and their variants have been utilized in wide range of applications in robotics, but also in 
other research domain, namely in bio-chemistry 
%Sampling-based motion planners has been intensively studied in robotics, but they have been applied also in other research fields.
%In bio-chemistry, the planners have been used to study
to study loop motions~\cite{cortes2004geometric},
%protein folding~\cite{amato2002using,raveh2009rapid,novinskaya2015improving,songPFintro},
protein folding~\cite{raveh2009rapid,novinskaya2015improving},
or for tunnel detection~\cite{vonasek2017tunnel}.

%or protein folding combined with ligand diffusion~\cite{cortes2010simulating}.
%rrt for molecular: \cite{al2012motion}
%survey \cite{gipson2012computational}
%Sampling-based planners have been used in various applications in robotics~\cite{elbanhawi2014sampling} and also  %,latombe1999motion} and also
%Solutions for tunnel detection using sampling-based methods however have not been discussed yet.
%Understanding of interactions between proteins and other small molecules is crucial in many research fields, including drug design and protein engineering. 
The problem of tunnel detection is to find a tunnel (pathways) leading to a specific place (called active site) inside the protein.
The knowledge about the existence of the tunnels and their properties (e.g., their bottleneck and chemical properties around the tunnel),
is important to understand interactions between proteins and other molecules (ligands).
To get the ligand to the active site, there has to be a transportation path connecting the protein outer environment with the active site.
The tunnels have to be wide enough and the physico-chemical properties of the surrounding amino acids have to be compatible with the ligand (Fig.~\ref{fig::motiv}).
%These interactions are highly influenced by the ability to transport the small molecule, called ligand, to the protein active site.
%The active site can be considered as a deeply buried inner cavity with the ability to interact with the incoming ligand.
The tunnels are usually detected using Voronoi diagrams~\cite{yaffe2008,caver3} assuming 
a spherical ligand (probe) and they are represented as a sequence of spheres.
However, as the ligands typically non-spherical and flexible,  it is difficult to estimate their traversability through the tunnels computed for only one sphere.

%Biochemists decide if a tunnel can be used to transport a given ligand mainly based on the tunnel length and bottleneck (i.e., the radius of the smallest sphere that forms the tunnel).
%This is used for example in protein engineering, where the task is to change selected properties of a protein, e.g., its stability under different outer conditions~\cite{Koudelakova2013}. % or its activity of the protein towards other molecules~\cite{Pavlova2009}.
%The design of suitable ligands causing the desired changes can be speeded up by detecting and analyzing the tunnels leading to the active sites.
%This can be achieved by detecting and studying so called tunnels in proteins which can serve as the transportation paths for the 
%ligand from the outside environment to the active site or vice versa. 
%Whereas the tunnel computation is already a well established research field, the simulation of ligand transportation through the detected tunnels is rather new.
%Ligands are typically of non-spherical shape, therefore it is difficult to estimate their traversability through tunnels computed for a spherical probe.
%The decision based only on spherical tunnels requires previous expertise in the domain and yet it may be imprecise.

To provide biochemists  a better insight into the behavior of non-spherical ligands, it is necessary to compute the trajectories of the ligand considering conformation changes.
The computation of trajectories for such ligands can be formulated as a motion planning problem.
RRT is a suitable candidate to solve this problem due to its ability to cope with many-DOF robots of arbitrary shapes.
This motion planning however require to cope with the narrow passage problem, as the proteins are dense structures and the
movements of the ligands are limited.
A narrow passage is a small region in the configuration space whose removal changes the connectivity of the space.
Due to its low volumes, the probability of sampling the narrow passages is low.

Consequently, many iterations are needed in order to put enough samples into the passages, which increases the computation time.
A proper distance metric in the configuration space is necessary in order to determine which samples should be connected together (in the case of PRM), or which node of the tree should be expanded (in the case of RRT).
Generally, it is not easy to combine the rotational and translational components of the configuration space.
Moreover, the selection of the metric is also influence by the shape of the objects (robots).

To cope with these issues, we propose to guide the sampling of the configuration space using the Voronoi-based tunnels.
The random samples are not generated from the whole configuration space, but mainly around the tunnel.
To increase movements of the ligands in the narrow tunnels, the free-space is dilated by 
shrinking the atom radii, similarly, e.g., to~\cite{cortes2005path,hsu06multilevel}.
The ligand can be considered as a multi-link kinematic chain and by changing the torsion angles, the flexibility can be modeled~\cite{songPFpath,cortes2005path}.
This however increases the dimension of the configuration space.
Instead, we model the flexibility using a fixed set of known conformations.

%Proteins are dense structures which limits movements of the ligand insider the tunnels, which leads to the narrow passage problem.
%To cope with this problem, we propose to utilize the principle of guided sampling~\cite{vonasek2009rrt,denny2016dynamic}.
%It helps to keep also potential solutions which do not fit to the geometric restrains but can be still feasible because of the physico-chemical properties.
%This strategy helps to overcome the limitations of the conformational discretation introduced by the rotamer approximation.

%The proposed method can be seen as an extension for the traditionally used tunnel detection tools.
%Besides trajectory computation, visualization of the results in important. %we also propose methods for their visualization.
%Our motivation is to help the biochemists to perform so called virtual screening, where they test the traversability of a ligand through a given tunnel.
%In order to predict the success of ligand traversability, the virtual screening performs hundreds of thousands of tests and checks if the ligand passed through the tunnel.
%The advantage of the proposed approach is that we can search the ligand path inside a specific tunnel which substantially decreases the computational time and resources required for the virtual screening.
%We further propose visualization techniques for presenting the results to biochemists.
%Therefore, methods for visualization of the results  are discussed.

\section{Related Work}

%The analysis of protein structure aiming to reveal the tunnels has been supported by different computational software tools which take the geometry of the protein as an input and explore the inner void space (e.g., CAVER 1.0~\cite{petrek2006caver} or MOLE~\cite{Petrek20071357}). 
%Early methods for tunnel detection utilized a discretized 3D grid, where each cell is considered as occupied or free depending
%on the presence of atoms of the protein~\cite{petrek2006caver,Petrek20071357}.

TODO some words about MD and why tunnels are predered

The analysis of ligand traversability to/from an active site is usually based on so called tunnels.
The tunnels are pathways computed for a single atom (spherical probe) using Voronio-diagrams~\cite{yaffe2008,caver3} or 
grid-based approaches~\cite{sehnal2013mole,petrek2006caver}.
The resulting tunnels are represented as a sequence of spheres and characterized by their bottleneck, length, curvature, and a list of surrounding residues, which are later used to estimate the interaction possibilities.
The main disadvantage of both grid-based and VD-based methods is that the shape of the ligand is not taken into account during the tunnel detection and it is therefore not easy to estimate if (and how) a ligand might traverse the tunnels.
In order to determine if a ligand can pass the tunnel, it is necessary to compute a trajectory considering the shape of the ligand.
This can be formulated as a motion planning problem in a high-dimensional configuration space $\C$ and solved using sampling-based planners.
In this case, the configuration space $\C$ has at least $6+n$ dimensions (3D rotation + 3D translation + additional degrees of freedom
caused by the ligand flexibility).

%Tunnels can be then searched using standard graph-search methods, like Dijkstra's algorithm.
%Besides, the grid can be used to identify other relevant properties like 
%pockets, cavities, or channels~\cite{sehnal2013mole,petrek2006caver}.
%One of the first grid-based approaches to the detection of tunnels in protein is the CAVER 1.0 algorithm by Pet\v{r}ek et al.~\cite{citeulike:6257975}.
%The obvious disadvantage of the grid-based methods is the high memory demand and their dependency on the grid resolution.
%Due to the high memory consumption, these methods are not suitable for tunnel detection in dynamic proteins and therefore they
%are used primarily for analysis of static molecules or individual snapshots of molecular dynamics.
%Currently the most widely used approach to tunnel detection is based on ordinary Voronoi diagrams (VD) or Weighted Voronoi Diagrams (WVD). %~\cite{caver3,yaffe2008}.
%The ordinary VD is computed on points representing centers of all atoms, without considering the radii of atoms.
%This may lead to detection of tunnels with incorrect bottlenecks. % i.e., incorrect radius of a smallest probe that can traverse the tunnels.
%To consider atoms with different radii, the weights of individual points are determined by the van der Walls radii of atoms in WVD.
%An alternative solution is to compute a non-weighted VD on an extended point set, where 
%each atom is approximated by several spheres with a small radius~\cite{yaffe2008,caver3}.
%VD-based methods are memory less demanding and also faster than the grid-based methods.
%The extension of VD-based methods to dynamic molecules requires to construct VD in the frames being analyzed and 
%finding correspondences between them.
%The existing approaches often use hierarchical clustering to match Voronoi vertices and edges from different frames, which is computationally demanding~\cite{lindow2012dynamic,caverDetails}.

%The protein tunnels computed for spherical probes contain valuable information about the protein void space.
%Based on the biochemical properties, the chemists can decide if a given tunnel is relevant and can be possibly used by a ligand to exit the protein.
%The biochemically relevant tunnels can be then analyzed separately.

%The configuration space $\C$ is formed by all possible configurations of the ligand in the tunnel, i.e., considering its rotation, translation, and possibly also other degrees of freedom responsible for the conformation changes.
%The dimension of the configuration space is given by the degrees of freedom (DOF) of the ligand, i.e., 6D for a rigid ligand and 6D + $n$ for a flexible
%ligand with $n$ DOFs.
%Sampling-based motion planning methods can be used to search this high-dimensional configuration space~\cite{Lav06}.

%Sampling-based motion planning methods are suitable for computing the trajectories of the ligand as they 
%can cope with many-DOF robots (objects) of arbitrary shapes.
%The flexibility of ligands (or even proteins) can be modeled using a multi-link kinematic chain, where torsional angles
%can change~\cite{songPFpath}.
%This is useful, e.g., in the protein folding studies~\cite{al2012motion,gipson2012computational,amato2002using,raveh2009rapid} or analysis of loop motions~\cite{cortes2004geometric}.
%This is useful, e.g., in the protein folding studies~\cite{al2012motion,gipson2012computational,amato2002using,raveh2009rapid,novinskaya2015improving,songPFintro} or analysis of loop motions~\cite{cortes2004geometric}.
%Sampling-based methods have also been used for 
% tunnel detection~\cite{vonasek2016application,vonasek2017tunnel} and
% exit pathway computation~\cite{cortes2010simulating,guieysse2008structure}.
%none of the existing work focuses on the trajectory generation in tunnels.
%However, not all conformations are feasible, which has to be checked using potential energy, which is time consuming.
%The idea of sampling-based motion planning is to randomly sample the configuration space $\C$ and classify the samples as free or non-free using collision detection.
%The free samples are stored in a roadmap (a graph structure), in which a path can be searched using standard graph-search methods.

Rapidly Exploring Random Tree (RRT)~\cite{lavalleRRT} is a single-query sampling-based motion planning method that 
incrementally builds a configuration tree $\T$ rooted at the initial configuration $\qinit$.
In each iteration of RRT, a random configuration $\qrand \in \C$ is generated and its nearest node $\qnear \in \T$ in the tree is found.
A new configuration $\qnew$ is constructed on the line connecting $\qnear$ and $\qrand$ in the distance $\varepsilon$ from $\qnear$.
If $\qnew$ is collision-free, it is added to the tree.
The algorithm terminates if the tree approaches the goal configuration close enough.

%rrt for molecular: \cite{al2012motion}
%survey \cite{gipson2012computational}
%Solutions for tunnel detection using sampling-based methods however have not been discussed yet.
%Sampling-based planners have been used in various applications in robotics~\cite{elbanhawi2014sampling} and also  %,latombe1999motion} and also
% to study proteins, e.g. in  
%loop motions~\cite{cortes2004geometric},
%protein folding~\cite{amato2002using,raveh2009rapid,novinskaya2015improving,songPFintro},
%protein folding~\cite{raveh2009rapid,novinskaya2015improving},
%or protein folding combined with ligand diffusion~\cite{cortes2010simulating}.

The well known issue of the sampling-based planners is the narrow passage problem.
%Sampling-based planners can provide poor performance in the presence of narrow passages, as many iterations
%are required in order to sample the narrow passages dense enough.
PRM-based planners can cope the narrow passage problem by increasing probability of sampling in given regions.
These regions can be estimated using workspace information such as by medial axis~\cite{amatoOBRRT,amato2002using,wilmarthMAPRM} 
or they can be identified e.g. based on number of collision-free samples around a given configurations~\cite{overmarsGauss,hsuBridge}.
The narrow passage are however located in the configuration space and it is generally not easy to estimate their location
based only on the features of the workspace~\cite{hannaWIS}.
%.g. based on amount of collision-free samples around a given sample~\cite{overmarsGauss} or e.g. based
%~\cite{hsuBridge,overmarsGauss}.

%In the Gaussian sampling strategy~\cite{overmarsGauss}, the samples are generated more frequently around obstacles.
%A random configuration $q_1\in\C$ is generated uniformly and another random configuration
%$q_2\in\C$ is drawn around using Normal distribution $N(q_1,\sigma^2)$.
%If both $q_1$ and $q_2$ are free or both are non-free, they are not added to the roadmap. 
%If only one of these configurations is free, then the free one is added to the roadmap. 
%The disadvantage of this strategy is the necessity to choose suitable value of the parameter $\sigma^2$, which depends
%on the used map and the shape of the robot.
%The Bridge-Test sampling~\cite{hsuBridge} employs the Gaussian strategy to generate two samples in order to compute their midpoint.
%The Gaussian strategy was improved in the Bridge-Test~\cite{hsuBridge}. % to generate samples inside a narrow passage.
%Two random configurations $q,q'\in\C$ are generated in same way as in the Gaussian strategy.
%The midpoint $p$ on a segment $q,q'$ is constructed.
%If the midpoint is collision-free and both end configurations are non-free, the midpoint lies in a narrow passage, and it is added to the roadmap.
%Both Gaussian and Bridge-Test strategies have to be combined with the uniform sampling in order to ensure sampling of free-regions~\cite{sun2005narrow}.
%Despite the simplicity of these two modifications, they were proven to significantly improve performance of PRM~\cite{hsuOnProb,geraertsRA,geraertsPRMA,wang10adaptive}.

RRT-based planners cannot cope with the narrow passage problem simply by increasing the probability of sampling in them, as the
growth of the configuration tree may be toward the random samples may be blocked by obstacles~\cite{vonasekphd}.
DD-RRT~\cite{yershovaDDRRT} limits the selection of nodes for expansion to a small ball. 
The radius of a ball is defaulty infinity, but if the node cannot be expanded, it is set to a predefied (small) value.
This small radius can be adapted according to the success rate of the expansion in ADD-RRT~\cite{jailletADRRT}.
To attract the tree towards a region, the random samples have to be generate there only if the tree can expand to the region.
This is the main idea of the guided sampling~\cite{vonasek2009rrt,denny2014marrt,denny2016dynamic}, where a the configurations
space is sampled with higher probability around a given list of configurations.
These guiding configurations can be computed e.g. as a path in the workspace.
Authors of~\cite{kardossRRTKK} proposed to place the guiding configurations between difficult and easy regions of the configuration space, but
they did not proposed any strategy to find the waypoints.

Besides increasing probability of sampling in certain regions, the growth of the tree towards the regions can be reached by retracting
the sampling to the regions.
%Another techniques to sample the narrow passages dense enough is to retract the random samples towards regions that are believed
%to be located in narrow passages~\cite{zhangRetraction,lee2012srrrt}.
In Retraction-RRT~\cite{zhangRetraction}, the random sample $\qrand$ is connected to the tree if the line segment $\qnear,\qrand$ is collision-free, otherwise, the retraction-step is performed.
The task of the retraction step is to find a contact configuration around $\qnear$ that minimizes the distance to $\qrand$.
The retraction step is computationally intensive, as the computation of the closest contact point leads to computation
of the generalized penetration depth, which has high complexity~\cite{zhang2008fast}.
Therefore, many algorithms uses a heuristic to compute the samples near boundary of the configuration space~\cite{amatoOBPRM,amatoOBRRT}.
The probability of sampling of the narrow passages can be increased by dilating the free-space, e.g. by shrinking the geometry of 
the robot~\cite{hsuOnProb} or the obstacles~\cite{bayazitIRC}.

The shrinking technique has been used also in~\cite{cortes2010simulating}, where exit pathways for a small flexible molecule are
computed using a RRT-based method.
To cope with many-DOF ligands, the RRT-ML variant~\cite{cortes2007mlrrt} is employed as the basic planner.
RRT-ML expands the tree primarily using those DOF that are essential for achieving motion of the ligand (i.e., rotation
and translation) and it employs the other DOFs (i.e., that are responsible for conformation changes) if they hinder the growth of the tree.


\section{Proposed method}


In this paper, we propose a modification of the RRT planner for computing trajectories through protein tunnels.
The input is a set of tunnels computed by Voronoi diagrams and the task is to find trajectories inside (or along) the tunnels
leading from outside the protein to the active site.

Contrary to~\cite{cortes2010simulating}, which find any exit pathway, we propose to compute the trajectories only around/inside each tunnel
separately.
This allows us to decrease the volume of the configuration space to be search, which increases the relative volume of the narrow passages.
To further cope with the narrow passage problem, the random samples are not generated in the whole configuration space, but only (mainly) around
the tunnel in a user-defined distance.
Therefore, the tunnels serve as a guiding path for the configuration tree~\cite{vonasek2009rrt}.

As the input data are snapshots of the proteins without ligands, it can be expected, that the tunnels are narrow therefore it would be difficult
to find trajectories for all ligands.
In reality, the tunnels adapt to the ligands (and vice versa), so even narrow tunnels can serve as transportation path for ligands larger than
the tunnel bottleneck TODO CITACE??.
To emulate this behavior, we shrink the atom radii of the ligand, similarly to~\cite{cortes2010simulating}.

The ligand flexibility is modeled using a predefined set of typical conformations, therefore the number of dimensions of the configuration space does not depend on the DOF of the ligand as in~\cite{cortes2010simulating}.
Library of conformations are used also in different types of calculations, e.g.~\cite{kellogg}. 
While classic 6D Euclidean metric is used in the nearest-neighbor search of RRT, we propose to use another metric during the expansion step.
This metric measures distances between two configurations as the nearest distance of two atoms. 
This enables the metric to consider the shape of the ligand and it support retraction of the ligand to the narrow passages.


%approximate the distance metric using nonlinear parametric regression model 
%\cite{palmieri2015Distance}
%true cost-to-go function implies solving two-point boundary problem which can be as expensive as solving the planning query itself.


%Motion planning for flexible ligands in protein tunnels brings two main issues: 
%the ligand flexibility increases the dimension of the configuration space, and the necessity to plan in protein tunnels
%leads to the narrow passage problem.
%A narrow passage is a region in the configuration space whose removal changes the connectivity of the free space~\cite{hannaWIS}.
%Narrow passages have smaller volume than other regions and it is therefore difficult to sample them dense enough using uniform distribution, that is used in the basic sampling-based planners.
%The presence of narrow passages in the configuration space, especially if they contain part of the solution, requires many iterations
%in order to put enough samples there and it consequently increases the planning time.
%Typical tunnels have bottlenecks smaller than $1.0$~\AA, so they can already be considered as narrow passages for ligands with more than two atoms.
%%The protein tunnels have the properties of the narrow passages, as they are very narrow
%To cope with the narrow passages, they have to be sampled more dense.
%For the family of RRT planners, it is useful to change the distribution of random samples according to the growth of the tree~\cite{vonasek2009rrt,vonasekphd,denny2016dynamic}.
%%about their 3D position, which is represented by the tunnel centerline.
%%The centerline can be used to guide the growth of the RRT tree through the configuration space~\cite{vonasek2009rrt,denny2016dynamic}.
%
%%The additional degrees of freedom required to model the flexibility increase the dimension of the configuration space.
%The kinematic chain representation used to model the ligand flexibility can generate all possible conformations but it also increases the dimension of the configuration space.
%Not all of conformations are however feasible and it is therefore necessary to verify the feasibility of a given conformation based on energy, which is time consuming.
%%RRT-based planners are sensitive to the employed metric, that should consider also the flexibility-related DOFs.
%An alternative solution is to employ a library of known conformations.
%The conformation changes of protein amino acids and ligand are represented by so called rotamers, stored in different rotamer libraries (e.g., Dunbrack library~\cite{dunbrack}).
%%Rotamers differ in the angles between their atoms.
%Possible rotamer conformations correspond to energetically and geometrically favorable positions.
%%This solution is common also in different types of calculations, e.g.~\cite{kellogg}. 





%is used also in other related tools like like MoMa-LigPath~\cite{cortes2005path}
%Such techniques have been used for motion planning of deformable objects, 
%e.g., in~\cite{frank2008efficient,bayazit2001ligand,alterovitz2008motion,lamiraux2001flexible,kavraki1998towards,gayle2005path}, 
%e.g., in~\cite{frank2008efficient,bayazit2001ligand,alterovitz2008motion,lamiraux2001flexible,gayle2005path}, 
%and for motion planning of rigid objects among 
%flexible obstacles~\cite{rodriguez2006planning,frank2008efficient,phillips2014representation}.

%Due to the interaction, the relative positions of the ligand's atoms can change.
% in cortes2005path:
%In this paper two kinds of large-amplitude motion are treated: protein loop conformational changes (involving pro-
%tein backbone flexibility) and ligand trajectories to deep active sites in proteins (involving ligand and protein side-chain flex-
%ibility). First studies performed using our two-stage approach (geometric search followed by energy refinements) show that,
%    compared to classical molecular modeling methods, quite   similar results can be obtained with a performance gain of
% several orders of magnitude. Furthermore, our results also  indicate that the geometric stage can provide highly valuable information to biologists.
%This technique is similar to generation of samples near the medial exist of the environment
%medial axis~\cite{wilmarthMAPRM,foskey01hybrid,guibas1999probabilistic,hoff2000interactive,yang2004adapting,amatoOBRRT}.
%or by guiding the growth of the tree around a general path in the workspace~\cite{vonasek2009rrt,denny2014marrt}.





\section{Traversability of Tunnels}

\subsection{Preliminaries}

Proteins and ligands are represented by the hard sphere model, where the radius of each sphere (atom) is given by its van der Waals radius.
The flexibility of a ligand is modeled using the set $\L$ of its conformations.
The conformations $L$ are used from a library (e.g.,~\cite{dunbrack}), or they can be prepared considering the potential energy.

%or downloaded from a library~\cite{dunbrack}.
%We assume that a conformation $l \in \L$ can be switched to all other conformations in $\L$.

%Another approach is to model the ligand flexibility using a predefined set of feasible conformations $\L$
%To decrease the computational burden, a set of feasible conformations $\L$ is prepared, e.g. using Rosseta tool.
%The conformations can be prepared using Molecular Dynamics software such as Rosetta.
%For each conformation $c\in \L$ an energy potential is also calculated.
%Let $\L_i$ denote the set of conformations that 
%The transition matrix $M$ defines the possible changes of conformations, i.e., $M(i,j)=1$ if the conformation $c_i \in C$
%can be switched to the confomration $c_j \in C$.

A protein tunnel is described by a sequence of collision-free spheres 
$T=( (c_1, r_1),\ldots,(c_n,r_n) )$, where $n$ denotes the number of spheres,
$c_i \in \R^3$ is their 3D position and $r_i > 0$ denotes the maximum collision-free radius of a sphere centered at $c_i$. 
The tunnels can be found by tunnel detection tools like CAVER 3.0~\cite{caver3}.

%from wiki:
%Multiple static structures experimentally determined for the same protein in different conformations are often used to emulate receptor flexibility.[19] Alternatively rotamer libraries of amino acid side chains that surround the binding cavity may be searched to generate alternate but energetically reasonable protein conformations.[20][21]
% 19 = \cite{totrov2008flexible}
% 20 = \cite{hartmann2009docking}
% 21 = \cite{taylor2003fds} 

%Conformations of the ligand may be generated in the absence of the receptor and subsequently docked[13] 
%or conformations may be generated on-the-fly in the presence of the receptor binding cavity,[14] 
%or with full rotational flexibility of every dihedral angle using fragment based docking.[15] 
%Force field energy evaluation are most often used to select energetically reasonable conformations,[16] but 
%knowledge-based methods have also been used.[17]
%13 = https://link.springer.com/article/10.1007/BF00123666

%The proteins are dense structures and the tunnels are typically narrow.
%Depending on the protein, the tunnel bottlenecks can be even less than $1$~\AA, which is too narrow for ligands with more than 2 atoms.
To enable motion of ligands in the narrow tunnels, the atomic radii of the ligand are scaled down by a factor $s, 0 < s \le 1$.
A discrete set of scales is used, i.e., $s \in \{\smin, \smin+\sdelta, \ldots, \smax\}$, where 
$\smin$ is the minimal allowed scale, $\smax=1$ is the maximal allowed scale and $\sdelta$ is the minimal difference between two scales.

A configuration of the ligand $q=(x,y,z,r_x,r_y,r_z,l,s)$  is described
by the 3D position $(x,y,z)$ of the reference point of the ligand (e.g., its geometric center), rotation around $x$, $y$, and $z$ axes,
index of the conformation $l\in \L$ and the actual scale $s$.
All possible configurations form the configuration space $\C$. 
A configuration is collision-free if none of the ligand atoms scaled by $s$ and placed at the
position defined by $q$ collides with the protein atoms.
%The set of all feasible configurations at scale $s$ is denoted $\CFD \subseteq \C$.


\subsection{Computing Initial Configuration}

The analysis of tunnel traversability is based on computation of multiple trajectories of the ligand inside the tunnel, which
requires to generate a set $\QI$ of collision-free initial configurations.
In this paper we assume that the ligand has to travel from the beginning to the end of the tunnel.
As the tunnels are computed for a spherical probe, the ligand may not fit exactly to the first sphere of the tunnel.
The initial configurations have to be searched around the first sphere of the tunnel.
To find a new initial configuration, a random sample $q$ is generated around $c_i$ in the distance $\RI$ (translation and rotation
 parts of $q$ are generated randomly, the scale is set to $\smin$ and the conformation index is set randomly).
If the sample $q$ is collision-free, it can be considered as a new starting configuration, so $q$ is added to $\QI$.
Similarly, a single goal configuration $\qgoal$ is found around the end of the tunnel. 
For each starting configuration $\qinit \in \QI$,  $K$ trajectories are created.
Each trajectory is computed using a modified RRT, which is introduced in the following section.


\subsection{Computing Single Trajectory of the Ligand}

The task of the trajectory computation is to find a trajectory for a ligand in the given tunnel. 
% which can be solved using RRT~\cite{lavalleRRT}.
Here, the original RRT is extended to cope with the specific requirements needed for the traversability of ligands.
First, the trajectory has to be found around the given tunnel, but deflection from the tunnel centerline is allowed.
%The deviation of the trajectory from the tunnel centerline is influenced by the parameter $\rv$.
%We assume, that the distance of the ligand from the tunnel centerline is at most $\rv$.
Due to this requirement, sampling process of RRT has to be adapted in order to follow the tunnel, and to prevent
construction of trajectories in the rest of the protein.
%The other modifications aim to cope with various scales of the ligand and multiple conformations,

The main loop of the proposed method is described in Alg.~\ref{alg::main}.
In each iteration, a random sample $\qrand$ is generated and its nearest node $\qnear\in\T$ in the tree is found.
The nearest-neighbor search between $\qrand$ and the tree is performed using the weighted 6D Euclidean metric considering
both 3D rotation and 3D translation.

To guide the growth of the tree through the tunnel, a moving virtual goal is used~\cite{vonasek2009rrt}.
The virtual goal $v, 1\le v \le n$, is the index of a sphere of the tunnel.
The random samples $\qrand$ are generated around the sphere $c_v \in T$ with the probability $\gb$, and from the whole $\C$ otherwise.
%with probability $1-\gb$ from $\C$, otherwise they are generated around the sphere $c_v \in T$.
After the tree reaches the sphere $c_v$, i.e., the distance of the tree to $c_v$ is
less than a predefined threshold $\dt$, the virtual goal is moved to the successor of the last sphere in the tunnel
that is reached by the tree (lines~\ref{alg::main:a}--\ref{alg::main:b} in Alg.~\ref{alg::main}).
Setting the virtual goal to this successor allows the tree to avoid such parts of the tunnels that are not traversable or reachable by the ligand.
This is necessary in the dense protein structures, where it is not always possible to exactly follow the tunnel.
The algorithm terminates after a predefined number of planning trials $\Imax$ or if the tree reaches
the last sphere in the tunnel, i.e., when $v = n$.

To generate the samples $\qrand$ around the virtual goal $v$, the translation part $(x,y,z)$ of $\qrand$ is generated
from $N(c_v,\Sigma)$, where $\Sigma$ is the diagonal matrix with diagonal entries equal to the parameter $\rv$, and the rotational
part of $\qrand$ is generated using techniques described in~\cite{kuffnerES}.
The other two parameters (ligand index $l$ and scale $s$) of $\qrand$ can be left zero, as these are not used in the employed
metric for the nearest-neighbor search.
The parameter $\rv$ influences the distribution of samples around the tunnel centerline. 
By setting $\rv$ to a small value, the planner attempts to find the trajectories inside the tunnel, while higher values
of $\rv$ cause  exploration of paths around the tunnel.
We propose to set this parameter to the average tunnel width.

\linesnumbered
\begin{algorithm}[h]
{\small
%\setstretch{0.88}
\caption{\label{alg::main}Main loop of the RRT planner}
\KwIn{
    tunnel $T=( (c_i, r_i) )$, $i=1,\ldots,n$, with spheres centers $c_i \in \R^3$ and radii $r_i$,
    initial configuration $\qinit$
%    goal configuration $\qgoal$,
}
\KwData{
   ligand conformations $\L$,
   scale limits $\smin, \smax$ and $\sdelta$
}
\KwOut{
    configuration tree $\T$\;
}
\hrule
$v = 1$; // index of the virtual goal\\
$iteration = 0$\;
\While{$iteration < \Imax$ {\bf and}  $v < n$}{
    \eIf{$rand() < \gb$}{
        $\qrand$ = random sample around actual virtual goal $c_v \in T$\;
    }{
        $\qrand$ = random sample from $\C$\;
    }
    $\qnear$ = nearest node in $\T$ towards $\qrand$\;
    expand($\qnear,\qrand$)\;
    \For{$i= n-1,n-2,\ldots,v+1,v$}{ \nllabel{alg::main:a}
        $d$ = nearest node in the tree towards sphere $c_i$\;
        \If{$d < \dt $}{
            $v = i+1$; // new virtual goal found\;
            {\bf  break}\;
        }
    } \nllabel{alg::main:b}
    $iteration = iteration+1$\;
}
\return $\T$\;
}
\end{algorithm}


The core of the proposed planner is the expansion procedure (Alg.~\ref{alg::expand}) which generates new collision-free nodes around $\qnear\in\T$.
For each ligand conformation $l \in \L$, the expansion procedure attempts to find a new collision-free configuration around $\qnear$ with a maximal scale.
First, the maximal scale $\smax$ is tested and $m$ random samples are generated around $\qrand$ and tested for collision.
The nearest collision-free sample towards $\qrand$ is selected and added to the tree.
If none of the tested samples is collision-free, the scale is reduced to $\smax-\sdelta$ and the search continues
until a collision-free sample is found or until the minimal reduced-scale $\smin$ is reached.
The random samples are generated similarly as in the case of $\qrand$ samples, only their translation
part is generated around $\qnear$.

The sampling-based methods are sensitive to the employed metric, especially if the objects are not symmetrical, which is the
case of the non-spherical ligands.
To consider the actual shape of the ligand (which is different in each conformation) and to support finding such configurations
that maximally approach $\qrand$, the distance between newly generated configurations and $\qrand$ is measured as the smallest
3D distance between an atom of the ligand placed at $q$ and the 3D position of $\qrand$ ($\da(q,\qrand)$ on line~\ref{alg::expand:a} in
Alg.~\ref{alg::expand}).
By computing the distance between the nearest atoms, the shape of the ligand is actually considered.
This metric supports retraction of the ligand towards $\qrand$.
%As the expansion routine attempts to find such sample that minimizes this distance to $\qrand$, it supports retraction of the ligand towards $\qrand$.


\begin{algorithm}[h]
{\small
%\setstretch{0.88}
%\setstretch{\straa}
\caption{\label{alg::expand}expand}
\KwIn{
   configuration $\qnear$ to be expanded,
   random configuration $\qrand$
}
\KwData{
   ligand conformations $\L$,
   scale limits $\smin, \smax$, and $\sdelta$,
   tree $\T$
}
%\KwOut{
%    set $S$ of collision-free configurations reachable from $\qnear$\;
%}
\hrule
\ForEach{$l \in \L$}{
    \ForEach{$s \in (\smax,\smax-\sdelta, \ldots, \smin+\sdelta, \smin)$}{
        $\qnew = \emptyset$; // empty configuration\\
        \For{$i = 1,\ldots,m$}{
            $q=\qnear$\;
            $q.position$ = random 3D position around $\qnear$\;
            $q.rotation$ = random 3D rotation\;
            $q.l = l$\; 
            $q.s = s$\;
            \If{isCollisionFree($q$)}{
                \If{$\qnew = \emptyset$ {\bf or} $\da(q, \qrand) < \da(\qnew,\qrand)$}{ \nllabel{alg::expand:a}
                    $\qnew = q$\;
                }
            } 
        }
        \If{$\qnew \ne \emptyset$} {
            $\T$.addNode($\qnew$)\;
            $\T$.addEdge($\qnear,\qnew$)\;
            {\bf break;} // go to next conformation
        }
    }
}
}
\end{algorithm}

The result of each planning trial is the tree $\T$ of collision-free configurations in which a path
between $\qinit$ (root of the tree) and $q'$ is found, where
 $q'$ is the nearest node in the tree towards $\qgoal$ (using 3D Euclidean metric). 
The path $P=(q_i), q_i \in \C$ is represented as a sequence of collision-free configurations.
The path is found in the tree even if the tree does not approach $\qgoal$ close enough.
Considering also these non-feasible solutions is necessary to evaluate difficult areas of the tunnels, e.g. bottlenecks.
The utilization of all computed paths for evaluation of tunnel difficulty is described in the next section.



\subsection{Visualization of Results}

%In order to fursupport the tunnel exploration process performed by biochemists.
The above defined characteristics can be shown even as a graph, or better, presented visually by mapping them using colors to the tunnel spheres.
 Alternatively, the surface representation can be used to visualize the tunnel.
In this case, a 3D point on the tunnel surface is colored according to the property value in its nearest tunnel part $i$,  i.e., such part $i$ whose center $c_i$ is the closest to the point among all tunnel parts.

The examples of the color mapping are shown in Fig.~\ref{fig:properties}.
The accessibility (Fig.~\ref{fig:properties}a) shows that more than the half of the tunnel is not accessible (red part of the tunnel).
The throughput shows (Fig.~\ref{fig:properties}b) that the only difficult is the part around the bottleneck (red color in Fig.~\ref{fig:properties}b), and the second half of the tunnel is also traversable.

\begin{figure}[h]
\centering
\begin{tabular}{ccc}
\includegraphics[width=0.3\textwidth]{fig/accessibility} &
\includegraphics[width=0.3\textwidth]{fig/throughput} &
\includegraphics[width=0.3\textwidth]{fig/ligand-scale} \\
  (a) & (b) & (c) \\                     
\end{tabular}
\caption{Tunnel properties mapped onto its surface. The tunnel begins at the top left corner.
(a) Accessibility --- green denotes the parts that were accessed by the ligand easily ($A(i) > 80$~\%), 
    while red denotes parts that were accessible  in $<10$~\% of cases.
(b) Throughput --- green denotes parts that are highly probable to be passed by the ligand ($T(i) > 80$~\%), 
    while red denotes those that were hard to pass ($T(i) < 20$~\%).
(c) Ligand scale --- the average scale of the ligand was \tylde 50\% in orange parts, \tylde 60\% in yellow parts, \tylde 80\% in yellow to green parts and \tylde 100\% in green parts.
\label{fig:properties}
}
\end{figure}



\def\cstart{\overline{c_{start}}}
\def\cend{\overline{c_{end}}}



\section{Experimental Verification}

The proposed approach was used to analyze tunnel traversability towards to active site of Haloalkane dehalogenase protein (PDB ID 1CQW).
The active site was defined using residues 38 (Asparagine), 102 (Histidine) and 103 (Aspartate).
Molecular dynamics of the protein was computed using TODO. 
First 100 frames were analyzed. 
In each frame, tunnels for spherical probe 0.9~\AA were dected using CAVER 3.0~\cite{caver3}. 
In most of the frames, two tunnels were detected.


%has been analyzed using the proposed approach.
%Three tunnels were detected using CAVER 3.0~\cite{caver3} for the spherical probe of radius 0.9~\AA (Fig.~\ref{fig::tunnel}a).
%The traversability was evaluated for 1-chlorpropan (denoted as $\LA$) with 11 atoms and 
% from David: m003 je "1-chlorpropan" a m004 "1-chlorbutan"
%and 1-chlorbutan (denoted as $\LB$) with 14 atoms.
%$\LA$ was represented by 12 conformations, and $\LB$ by a set of $114$ conformations.
%Examples of three conformations of $\LA$ and $\LB$ are depicted in Fig.~\ref{fig::tunnel}b.
%Further information about the experiments can be found at %{\url{http://mrs.felk.cvut.cz/isrr2017}}.

%Both tested ligands have more than 10 atoms and therefore they cannot fit into the tunnel of with 0.9~\AA, so
%the radii of ligands were scaled down. 
Four different scaling-down factors were used: $\smin=\{0.5,0.6,0.7,0.8\}$.
For each scale, 20 collision-free initial configurations were found in the $\RI=2$\AA\ radius around the first sphere of the tunnel (outside the protein).
For each ligand, each minimal scale $\smin$ and each initial configuration, 10 trajectories were computed using the proposed planner.
The parameters of the planner were: $\Imax=10,000$, $m=50$, $\gb=0.9$, $\dt=1.5$\AA, $\rv=2$~\AA.

For each initial configuration, the success rate is computed as the ratio of trajectories that reached the end of the tunnel (to the distance 
 $\rv=2$~\AA or less) over 100 trials.

\def\tmpa{0.13\textwidth}

%\begin{figure}
%\centering
%\begin{tabular}{cc}
%\includegraphics[width=0.3\textwidth]{fig/t05proteintunnels} &
%\includegraphics[width=0.45\textwidth]{fig/ligAll}  \\
%(a) & (b)                        
%\end{tabular}                       
%\caption{\label{fig::tunnel}
%    (a) The protein 1CQW visualized using the cartoon representation (gray) with 
%        three tunnels (red, green, blue) detected by CAVER 3.0.
%        The first tunnel is depicted red.
%%        its highest ranked tunnel (green) which was detected using CAVER 3.0.
%    (b) Examples of three conformations of $\LA$ (top) and $\LB$ (bottom)
%}
%\end{figure}
%


The results are shown in Tab.~\ref{tab::main}.


\begin{table}
\centering
\caption{\label{tab::main}Traversability of the tunnels}
\small
\renewcommand{\tabcolsep}{1pt}
{\small
\input graph.tex
}
\end{table}





\section{Discussion}

The computed trajectories cannot be considered as `real', they should rather be considered as a hint for the biochemists.
One of the reason is that the trajectories are computed inside a static protein and static tunnels.
Real proteins are dynamic structures and consequently also the tunnels are dynamic: they move, merge or even disappear due to motions of protein atoms.
Contrary, usually static tunnels are used in protein engineering.
Despite such a simplification, researches often used information about the static tunnels  to estimate the possibility of chemical reactions
at the active sites.

The second limitation is caused by the utilization of scaled-down ligands.
The protein atoms fill the internal void space therefore the tunnels identified inside proteins without ligand tend
to be very narrow.
To enable at least some motion of the ligands inside such narrow tunnels, the scaling-down of the ligand atoms is necessary.
It is used also in other related tool, e.g.~\cite{cortes2005path}.

Due to these reasons, the computed trajectories can be considered either as too optimistic
(e.g. because they are computed on a wide tunnel that can be in fact closed due to molecular dynamics),
or too pessimistic (e.g. the trajectory is not found because of narrow passage around the tunnel, which can be 
opened due to the molecular dynamics).
Despite these limitations, testing the tunnel traversability using motion planning technique can provide chemists more information
than the simple bottleneck radius, which is used nowadays.
For example, possible detours from the tunnel can be identified, which indicates that even a tunnel with a small
bottleneck can be traversed (Fig.~\ref{fig::detail}).

%Most related works, e.g. those based on motion planning, are usually focused on protein folding, or finding arbitrary pathways
%from proteins~\cite{cortes2010simulating,guieysse2008structure}.

The proposed planner is very simple and it can be improved in many ways: by using a less aggressive scaling-down technique (in order
to obtain better trajectories) or by using only a subset of conformations and a better expansion (in order to speed it up).
Having a better motion planner (in the term of the speed or even success rate) however does not automatically mean that the resulting trajectories will help the biochemists more.
The presented work shows possible advantages of the analysis based on ligand trajectories.
By comparing the planned trajectories with trajectories observed in MD simulations, thresholds for the success rate or minimal tunnel throughput can be set, which will help biochemists to make decisions.



%The runtime of the planner is currently given mainly by the collision-detection.
%As the ligand is assumed to travel only along the tunnel centerline (in the distance defined by $\rv$),
%only atoms located in the vicinity of the tunnel can be considered for collision detection.
%Fast collision-detection can be achieved either using fast hierarchical data structures as OBB (implemented, e.g., in~\cite{ozcollide})
%    or using collision-detection system developed directly for proteins~\cite{ruiz2005biocd}.
%We assume that a conformation $l \in \L$ can be switched to all other conformations in $\L$.

%parmas:
%$\RI$
%$\rv$
%$\dt$
%The presented expansion procedure favors the samples with maximal scale (they are tested first), the trajectory contains samples with higher scales at the easy part of the tunnels (e.i., wide areas), and the smaller scales are used only if the ligand has to fit into a narrow space.



\section{Conclusion }
The protein tunnels are transporting pathways for ligands to the  active sites.
In this paper, we propose to analyze the traversability in the tunnels using motion planning, that can consider both shape
of the protein and ligand.
We proposed modifications to the RRT method to compute trajectories for a ligand represented by a library of common conformations.
The corresponding configuration space is sampled using the guided sampling, where a virtual sphere moves along the tunnel and attracts growth
of the tree towards it.
To enable trajectory computation in narrow tunnels, atomic radii of the ligand are scaled down.
Computed trajectories allows us to evaluate the traversability of the tunnels using
accessibility, throughput and also by visualizing typical pathways found by the planner.
These properties can help to better understand importance of the tunnels.


\bibliographystyle{plain}
\bibliography{paper}

\end{document}



% ============  unused ideas =========================

%neni treba \cite{cheng01reducing}
%Using a library of predefined conformations has been used also in other tasks, e.g.~\cite{kellogg}.
%This shrinking technique is also used in the MoMa-LigPath tool~\cite{cortes2005path}, which utilizes RRT to find pathways for unbinding a ligand  from a protein.
%It is necessary to use an apporpriated distance metric in the configuration space.
%Generally, it is hard to combine the translational and rotational components in the configuration space.
%In~\cite{zhangRetraction}, the DISP metric~\cite{zhang2008fast}.
%energetic constraints are translated into geometric ones by considering a steric model of the molecules and applying collision detection
%\cite{deAngulo2005biocd}
%First the biredge test is applied to identify regions aeround narrow passages, optimization-based retraction operatoin sleectively only at those regions selective rrt~\cite{lee2012srrrt}
%Free-space dilation is an effective aproach for narrow passage sampling~\cite{hsu06multilevel}. 
%It is necessary to define proper method for dilating and determin the amount of dilation.
%\cite{saha2005finding}
%DISP metric \cite{zhang2008fast}


\subsection{Traversability Characteristics}

For each initial configuration $\qinit \in \QI$, $K$ trajectories are computed, which results in the set of $K |\QI|$ trajectories.
All these trajectories are used to compute the following properties of the tunnel $T$.
A trajectory $P$ reaches the tunnel sphere $c_i \in T$, if the 3D Euclidean distance of the 
nearest configuration $q \in P$ towards $c_i$ is less than $r_i$ (radius of the sphere $c_i$).
Let $N(i)$ denote the number of trajectories that reached $i-$th sphere of the tunnel, $i=1,\ldots,n$.
Three basic characteristics of the tunnel are computed from the trajectories: accessibility, throughput, and the scaling factor.

The accessibility $A(i)=N(i)/N$ of the sphere $i$ is the probability of reaching the sphere $i$, where $N=K|\QI|$ is the total number of trajectories.
The accessibility shows how probable it is to pass the tunnel up to the sphere $i$.
Obviously, the most important is $A(n)$ of the last sphere of the tunnel, which can be considered as the overall difficulty of the tunnel.
The  ligand passage may however be strongly affected by the first bottleneck, so the parts of the tunnel located behind
the bottleneck have low accessibility.

The throughput $T(i)$ is the ratio of trajectories that passed sphere $i$ (i.e., visited sphere $i+1$) and reached the sphere $i$,
i.e., $T(i) = N(i+1) / N(i)$. 
The throughput is not computed for the last sphere ($i=N$).
The throughput shows a local accessibility of tunnel parts and it can be used to detect places where most of the trajectories ends.

The proposed planner is allowed to scale down the radii of ligand spheres up to the allowed scale $\smin$. 
It can be expected that narrower parts of the tunnel are more often passed with a more scaled-down ligand than
the wider parts.
Ligand scale $L(i)$ at the tunnel sphere $i$ is the average scale of ligand that reaches the sphere $i$.


\begin{figure}
\centering
{
\renewcommand{\tabcolsep}{-5pt}
\begin{tabular}{ccc}
\includegraphics[width=0.38\textwidth]{fig/t05proteinBottle2T} & 
\includegraphics[width=0.38\textwidth]{fig/t04goodT} & 
\includegraphics[width=0.38\textwidth]{fig/t05goodT} \\
\includegraphics[width=0.38\textwidth]{fig/t05thpT} &
\includegraphics[width=0.38\textwidth]{fig/t04badT} &
\includegraphics[width=0.38\textwidth]{fig/t05badT} \\ 
(a) & (b) & (c)                       
\end{tabular}                       
\caption{\label{fig::tunnel2}
    (a) Classic bottleneck for spherical probe (top) and visualization of throughput (bottom) for the ligand $\LA$.
    (b) Successful (green, top) trajectories that reached end points of the tunnels,
        and unsuccessful ones (red, bottom) for $\smin=0.4$.
    (c) Successful and unsuccessful trajectories for $\smin=0.5$.
}
}
\end{figure}


\begin{figure}
\centering
{
\renewcommand{\tabcolsep}{0pt}
\begin{tabular}{ccc}
\includegraphics[width=0.325\textwidth]{fig/tunne4ac}& 
\includegraphics[width=0.325\textwidth]{fig/tunne4bc} & 
\includegraphics[width=0.325\textwidth]{fig/tunne4cc} \\
(a) & (b) & (c)                        
\end{tabular}
}
\caption{\label{fig::detail}
    Detailed view to alternative pathway around the third tunnel in 1CQW.
    The successful trajectories are in green (a) the unsuccessful in red (b).
    (c) Shows visualization of protein atoms around the tunnel.
}
\end{figure}


Besides the color mapping, it is also necessary to show the computed trajectories.
%This is useful, for example, to find out in which part of the tunnel the trajectories detour.
Simple visualization of all trajectories could however be too slow for an interactive work.
Therefore, the trajectories are first clustered and then only the clusters are visualized.
Due to different lengths of the trajectories, they are first converted to a normalized form.
Let $\cstart$ represent the average starting position of all trajectories and let $d_{max}$ be the 3D Euclidean distance
of the most distant configuration from $\cstart$ among all trajectories.
A set of $M$ spheres centered at $\cstart$ are created with radii $r'_i=i {d_{max} \over M}$, where $i=0,\ldots, M-1$.
The trajectory $P=(q_1,\ldots,q_n)$ of length $n$ is represented by the normalized vector $v=(x_1,\ldots,x_M)$ of length $M$,
where $x_i$ is the 3D position of the nearest configuration $q \in P$ to the surface of the $i$-th sphere with the radius $r'_i$.
The distance between two normalized trajectories $v_i$ and $v_j$ is defined as
$d(v_i, v_j) = \frac{1}{M} \sum_{1 \leq k \leq N} |x_k^i x_k^j|$.
This distance is used in the UPGMA clustering technique~\cite{sokal1958statistical}.
The trajectories can be visualized using a representative of each cluster.
The number of trajectories in each cluster is represented by the width of the polyline (Fig.~\ref{fig:trajectories}).
%In this manner, we are able to convey the information about all different trajectories together (see Fig.~\ref{fig:trajectories} right).


\begin{figure}
\centering
\begin{tabular}{cc}
\includegraphics[width=0.4\textwidth]{fig/trajectories-all} &
\includegraphics[width=0.4\textwidth]{fig/trajectories-clustered-21} \\
(a) & (b) \\                       
\end{tabular}
\caption{Visualization of the trajectories. The tunnel begins in the top left corner.
(a) All trajectories ($\sim5000$) colored according to whether they reached the end of the tunnel (green) or not (red). 
(b) Visualization using clusters of trajectories. 
%Trajectories clustered according to their proximity colored by distinct colors.
%The different line widths represent the number of trajectories in a cluster.
\label{fig:trajectories}
}
\end{figure}



\begin{table}
\centering
\caption{\label{tab::rrt}
    Runtime and success ratio for ligand $\LA$ (left) and $\LB$ (right).
}
{
\small
\renewcommand{\tabcolsep}{1pt}
\input table.003.tex
\hskip 4pt   
\input table.004.tex
}
\end{table}


Trajectories for $\LA$ in all detected tunnels were classified as successful if they reached the last sphere of the tunnel
to distance $\dt=2$~\AA and they were considered unsuccessful otherwise.
The throughput computed from the trajectories shows that the tunnels are difficult not only around bottlenecks, but
also in other places.
The comparison of the classic bottleneck (i.e., measured by the radius of spherical probe) and throughput is depicted in Fig.~\ref{fig::tunnel2}a.
The trajectories for $\smin=0.4$ are depicted in Fig.~\ref{fig::tunnel2}b and for $\smin=0.5$ in Fig.~\ref{fig::tunnel2}c.
The successful trajectories for $\smin=0.4$ reveal that the end of the tunnel No. 3 can be approached by two different pathways (one in the tunnel and another one outside the tunnel).
The detail is depicted in Fig.~\ref{fig::detail}.
Despite the low bottleneck of this tunnel, the ligand may reach its end using the alternative pathway. 



